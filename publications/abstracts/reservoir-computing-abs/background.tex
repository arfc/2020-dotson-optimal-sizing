\section{Background}
Variability has been the primary drawback for renewable energy sources like
wind turbines, solar PV, and solar concentrators, since their inception. This
flaw has become more pronounced as renewable penetration on the electricity
grid increased in recent years. Forecasting electricity production from
renewable sources is therefore important for successful management of power
systems \cite{kobylinski_high-resolution_2020}. Recent studies have applied
\glspl{ann}, specifically multi-layer perceptrons, to the task of net load
forecasting \cite{kobylinski_high-resolution_2020,dutta_load_2017,lee_development_2016}. These studies made short
term forecasts of 4-6 hours. Nuclear plants
need accurate forecasts further ahead to facilitate relaxed load following.
This study will be the first to apply \glspl{ESN} to the task of net load
prediction.

The University of Illinois at Urbana-Champaign is an ideal model system for
this work because of its diverse energy mix. Previous work has been done to
characterize this energy grid and optimize the size of a nuclear reactor
\cite{dotson_optimal_2020}. Due to the degree of wind penetration, the
University is sometimes
forced to sell electricity back to the grid operator, MISO, at a loss because
of overproduction
from wind energy. Thus, a reliable prediction of electricity production from
wind and other variable sources will reduce the likelihood of these events.

\glspl{ESN}, a flavor of reservoir computing, are a modern
machine learning algorithm that enables accurate short
to medium term predictions. Pathak et. al used an \gls{ESN} to predict the
evolution of a chaotic system, a laminar flame front, up to seven Lyapunov
times in the future \cite{pathak_model-free_2018, wikner_combining_2020}. A
Lyapunov time simply measures the timescale at which chaos makes initial
predictions useless. The effect of chaos typically overwhelms conventional
predictions after a single Lyapunov time, by definition.
The Lyapunov time for a weather system is on the order of a few days but
depends on the regional environment. \glspl{ESN} have also been used to
forecast multivariate time series \cite{bianchi_reservoir_2020}. Echo state
networks are unique among neural
networks in their ease of implementation and training speed. This is owed to its
sparse network architecture \cite{pathak_model-free_2018,
wikner_combining_2020, vannitsem_predictability_2017}. However,
their simplicity is balanced by the need for carefully chosen hyperparameters
for the desired task \cite{lukosevicius_practical_2012}.
Combining accurate demand and
renewable energy predictions will enable an artificially intelligent reactor
operator to adjust power in a relaxed manner.
